% This is samplepaper.tex, a sample chapter demonstrating the
% LLNCS macro package for Springer Computer Science proceedings;
% Version 2.20 of 2017/10/04
%
\documentclass[runningheads]{llncs}
%
\usepackage{graphicx}
\usepackage{subfig}

% For todo notes (workaround for 2-column using marginnote from: https://tex.stackexchange.com/questions/52680/how-can-i-make-todo-comments-when-using-the-multicol-package)
\usepackage[textsize=tiny, textwidth=2.0cm]{todonotes}


\begin{document}
%
\title{State Estimation and Data Assimilation for an Agent-Based Model using a Probabilistic Framework 
\thanks{This work was supported by a European Research Council (ERC) Starting Grant [number 757455], a UK Economic and Social Research Council (ESRC) Future Research Leaders grant [number ES/L009900/1], an ESRC-Alan Turing Fellowship [ES/R007918/1] and through an internship funded by the UK Leeds Institute for Data Analytics (LIDA).}}

%
\titlerunning{Probabilistic estimation of ABMs}
% If the paper title is too long for the running head, you can set
% an abbreviated paper title here
%

\author{Nick Malleson\inst{1,3}\orcidID{0000-0002-6977-0615} \and
Luke Archer\inst{3} \and
Jonathan A. Ward\inst{2}\orcidID{0000-0003-3726-9217} \and
Alison Heppenstall\inst{1,3}\orcidID{0000-0002-0663-3437} \and
IMPROBABLE?\inst{4}
%Daniel Tang\inst{4}  \and
%Jonathan Coello\inst{4}
}
%

\authorrunning{Malleson et al.}
% First names are abbreviated in the running head.
% If there are more than two authors, 'et al.' is used.
%
\institute{
School of Geography, University of Leeds, LS2 9JT, UK \\
\url{http://geog.leeds.ac.uk/} \\
\email{n.s.malleson@leeds.ac.uk} 
 \and
School of Mathematics, University of Leeds, LS2 9JT, UK \\
\url{http://maths.leeds.ac.uk} 
\and
Leeds Institute for Data Analytics (LIDA), University of Leeds, LS2 9JT, UK \\
\url{http://lida.leeds.ac.uk} \and
Improbable, 30 Farringdon Road, London, EC1M 3HE, UK \\
\url{http://www.improbable.io}
}
%
\maketitle              % typeset the header of the contribution
%
\begin{abstract}

XXXXX NICK TO WRITE ABSTRACT

\keywords{Agent-based modelling \and Probabilistic programming \and Uncertainty \and Data assimilation \and State estimation \and Bayesian inference }
\end{abstract}
%
%
%
 
 \newpage
\pagenumbering{arabic} % Reset the page number as everything up to now was a title & abstract

%
%
% ***************** Introduction *****************
%
%

\section{Introduction and Objectives}




\begin{itemize}
\item Aim: Experiment with probabilistic modelling and probabilistic programming as a means of performing state estimation and data assimilation on a agent-based model.
\item Method: Apply the framework to an ABM with decreasing amounts of information about the truth:
	\begin{enumerate}
	\item Full information about all agents (with a bit of noise)
	\item Information about only some agents (i.e. we're tracking a few individuals)\todo{Luke is this realistic by Feb 22?}
	\item Only aggregate information (This will be future work).
	\end{enumerate}
\end{itemize}

\todo[inline]{Note that this is not calibration - it's state estimation through data assimilation}

%
%
% ***************** Background *****************
%
%
\section{Background}

$ $ % (this is because a \todo straight after a \section confuses latex)

\todo[nolist, inline]{How this work fits in to the wider data assimilation schema (is it `nudging'? and how it compares to traditional data assimilation. Basically a very, very brief literature review.}

\todo[nolist, inline]{Outline what probabilistic programming is, and what Keanu is.}




\section{An Example Agent-Based Model: \textit{StationSim}}

$ $ % (this is because a \todo straight after a \section confuses latex)

\todo[nolist, inline]{NM: Briefly outline station sim to show that it has some of the normal characteristics of an ABM (one paragraph).}


%
%
% ***************** Method  *****************
%
%

\section{Data Assimilation Framework}

$ $ % (this is because a \todo straight after a \section confuses latex)

\todo[nolist, inline]{Explain the basic framework here, e.g. number of iterations, number of windows, calculating the posterior for the state, etc. We apply the same framework to the simple model and station sim.}

\section{Results}

\subsection{Full Knowledge of the System}

$ $ % (this is because a \todo straight after a \section confuses latex)

\todo[inline, nolist]{Experiments when the probabilistic model has full knowledge of the system}

\subsection{Knowledge of Only Some Agents}

$ $ % (this is because a \todo straight after a \section confuses latex)

\todo[inline, nolist]{Experiments when we only give the probabilistic model access to partial information in the state vector (i.e. only a few agents)}


%
%
% ***************** Conclusion *****************
%
%
\section{Conclusions}

XXXX Conclusions

\subsection*{Link to ABMUS Workshop Themes}

This paper contributes to the challenge set out in the workshop theme by developing methods that support ``trusted models that can be used by industry and governments to enhance decision-making, and that can incorporate real (and real-time) data sets in a meaningful way''. Without a reliable means of incorporating real-time data into urban models, their use in forecasting will always be limited to providing likely outcomes based on historical data.

(This is just to make the referencing work initially - can be deleted later \cite{epstein_growing_1996} ).


\bibliographystyle{splncs04}
\bibliography{2019-ABMUS-KeanuProbabilistic}

\end{document}
