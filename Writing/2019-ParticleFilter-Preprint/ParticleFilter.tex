\documentclass{article}

\usepackage{arxiv}


\usepackage{graphicx}
%\usepackage{subfig}
\usepackage{url}
\usepackage{hyperref}
\usepackage[numbers,sort&compress]{natbib}
\usepackage{amsmath}
\usepackage{nicefrac}       % compact symbols for 1/2, etc.
\usepackage[utf8]{inputenc}
\usepackage[T1]{fontenc}
\usepackage{appendix}
\usepackage{subcaption} % for sub figures

% For todo notes (workaround for 2-column using marginnote from: https://tex.stackexchange.com/questions/52680/how-can-i-make-todo-comments-when-using-the-multicol-package)
%\usepackage[textsize=tiny, textwidth=2.0cm]{todonotes}

% The aim of the paper (so that if it changes slightly this is reflected everywhere)
% Call simply by using '\aim'
\newcommand{\aim}{quantify the conditions under which a typical particle filter is able to reliably estimate the `true' state of an underlying pedestrian system through the combination of a modelled state estimate, produced using an agent-based model, and observational data}


%
\title{Simulating Crowds in Real Time with Agent-Based Modelling and a Particle Filter
\thanks{This project has received funding from the European Research Council (ERC) under the European Union’s Horizon 2020 research and innovation programme (grant agreement No. 757455), through a UK Economic and Social Research Council (ESRC) Future Research Leaders grant [number ES/L009900/1], and through an internship funded by the UK Leeds Institute for Data Analytics (LIDA).}}

\author{
  Nick Malleson\thanks{Corresponding author}\\
  School of Geography\\
  University of Leeds\\
  Leeds, LS2 9JT, UK\\
  \textit{n.s.malleson@leeds.ac.uk}\\
 \And 
 Kevin Minors\\
  Leeds Institute for Data Analytics\\
  University of Leeds\\
  Leeds, LS2 9JT, UK \\
   \And 
 Le-Minh Kieu\\
  Leeds Institute for Data Analytics\\
  University of Leeds\\
  Leeds, LS2 9JT, UK \\
   \And 
 Jonathan A. Ward\\
  School of Mathematics\\
  University of Leeds\\
  Leeds, LS2 9JT, UK \\
   \And 
 Andrew A. West\\
  School of Geography\\
  University of Leeds\\
  Leeds, LS2 9JT, UK \\
   \And 
 Alison Heppenstall\\
  %School of Geography\\
  %University of Leeds\\
  %Leeds, LS2 9JT, UK \\
  Alan Turing Institute\\
  British Library\\
  London NW1 2DB \\
 } 

%\institute{
%School of Geography, University of Leeds, LS2 9JT, UK \\
%\url{http://geog.leeds.ac.uk/} \\
%\email{n.s.malleson@leeds.ac.uk} 
% \and
%School of Mathematics, University of Leeds, LS2 9JT, UK \\
%\url{http://maths.leeds.ac.uk} 
%\and
%Leeds Institute for Data Analytics (LIDA), University of Leeds, LS2 9JT, UK \\
%\url{http://lida.leeds.ac.uk} \and
%Improbable, 30 Farringdon Road, London, EC1M 3HE, UK \\
%\url{http://www.improbable.io}
%}
%

\begin{document}
\maketitle              % typeset the header of the contribution
%
\begin{abstract}
Agent-based modelling is a valuable approach for systems whose behaviour is driven by the interactions between distinct entities. They have shown particular promise as a means of modelling crowds of people in streets, public transport terminals, stadiums, etc. However, the methodology faces a fundamental difficulty: there are no established mechanisms for dynamically incorporating \textit{real-time} data into models. This limits simulations that are inherently dynamic, such as pedestrian movements, to scenario testing of, for example, the potential impacts of new architectural configurations on movements. This paper begins to address this fundamental gap by demonstrating how a particle filter could be used to incorporate real data into an agent-based model of pedestrian movements at run time. The experiments show that it is indeed possible to use a particle filter to perform online (real time) model optimisation.  However, as the number of agents increases, the number of individual particles (and hence the computational complexity) required increases exponentially. By laying the groundwork for the real-time simulation of crowd movements, this paper has implications for the management of complex environments (both nationally and internationally) such as transportation hubs, hospitals, shopping centres, etc.

\end{abstract}

\keywords{Agent-based modelling \and Particle Filter \and Data assimilation \and Crowd simulation}

% !TEX root = ParticleFilter.tex
\section{Introduction\label{introduction}}

Agent-based modelling is a form of computer simulation that is well suited to modelling human systems~\citep{bonabeau_agent_2002,  farmer_economy_2009}. In recent years it has emerged as an important tool for decision makers who need to base their decisions on the behaviour of crowds of people~\citep{henein_agentbased_2005}. Such models, that simulate the behaviour of synthetic individual people (`agents'), have been proven to be useful as tools to experiment with strategies for humanitarian assistance \citep{crooks_gis_2013}, emergency evacuations \citep{ren_agentbased_2009, schoenharl_design_2011}, religious festivals~\citep{zainuddin_simulating_2009}, crowd stampedes~\citep{helbing_simulating_2000} etc. Although many agent-based crowd simulations have been developed, there is a fundamental methodological difficulty that modellers face: there are no established mechanisms for incorporating real-time data into simulations \citep{lloyd_exploring_2016, wang_data_2015, ward_dynamic_2016}. Models are typically calibrated once, using historical data, and then projected forward in time to make a prediction independently of any new data that might arise. Although this makes them effective at analysing scenarios to create information that can be useful in the design of crowd management policies, it means that they cannot currently be used to simulate real crowd systems \textit{in real time}. Without knowledge of the current state of a system it is difficult to decide on the most appropriate management plan for emerging situations.

Fortunately, methods do exist to reliably incorporate emerging data into models. \textit{Data assimilation} (DA) is a technique that has been widely used in fields such as meteorology, hydrology and oceanography, and is one of the main reasons that weather forecasts have improved so substantially in recent decades \citep{kalnay_atmospheric_2003}. Broadly, DA refers to a suite of techniques that allow observational data from the real world to be incorporated into models \citep{lewis_dynamic_2006}. This makes it possible to more accurately represent the current state of the system, and therefore reduce the uncertainty in future predictions.

It is important to note the differences between the data assimilation approach used here and that of typical agent-based parameter estimation / calibration. The field of optimisation -- finding suitable estimates for the parameters of algorithms -- has (and continues to be) an extremely well-researched field that agent-based modellers often draw on. For example, agent-based models regularly make use of sampling methods, such as Latin Hypercube sampling \citep{thiele_facilitating_2014} or evolutionary / heuristic optimisation algorithms such as simulated annealing~\citep{pennisi_optimal_2008}, genetic algorithms,~\citep{heppenstall_genetic_2007}, and approximate Bayesian computation~\citep{grazzini_bayesian_2017}. There are also new software tools becoming available to support advanced parameter exploration \citep{ozik_extremescale_2018}. It is, however, worth noting that in most cases agent-based models are not calibrated to quantitative data~\citep{thiele_facilitating_2014}.  In the cases where parameter estimation does take place, it is typically performed as a single calibration step in a waterfall-style development process -- e.g. design, implementation, calibration, validation. Although there are some more recent studies that do attempt to re-calibrate model parameters dynamically during runtime \citep[e.g.][]{oloo_adaptive_2017a, oloo_predicting_2018} there is another, more fundamental, difference to typical parameter optimisation (be it static or dynamic) and the data assimilation approach. 

Even if optimal model parameters have been found, there will usually be a degree of uncertainty in the model. With crowd simulations, for example, it is impossible to know \textit{exactly} how individuals will behave in a given situation -- will someone turn left or right given two competing options? -- nor can individual parameters such as walking speed ever be known exactly. Data assimilation algorithms use `state estimation' to calculate the difference between the model and the `true' state of the underlying system at runtime. They are then able to adjust the current model state in order to constrain a model's continued evolution against the real world~\citep{ward_dynamic_2016}. Although it is possible to re-calibrate models dynamically during runtime (e.g. \citep{oloo_adaptive_2017a}), this would not reduce the natural uncertainty that arises as stochastic models evolve. 

This paper is part of a wider programme of work\footnote{\url{http://dust.leeds.ac.uk/}} whose main aim is to develop data assimilation methods that can be used in agent-based modelling. The software codes that underpin the work discussed here are available in full from the project code repository; see Appendix~\ref{appendix:code}. The work here focuses on one particular system -- that of pedestrian movements -- and one particular method -- the particle filter. A particle filter is a brute force Bayesian state estimation method whose goal is to estimate the `true' state of a system, obtained by combining a model estimate with observational data, using an ensemble of model instances called particles. When observational data become available, the algorithm identifies those model instances (particles) whose state is closest to that of the observational data, and then re-samples particles based on this distance. It is worth noting that once an accurate estimate of the \textit{current} state of the system has been calculated, predictions of future states should be much more reliable -- c.f. the substantial improvements in weather forecasting that have come about as a result of modern data assimilation methods~\citep{kalnay_atmospheric_2003}. Predicting future system states is beyond the scope of this paper however.

The overall aim of the paper is to: \begin{quote}\textit{\aim}.\end{quote} This will be achieved through a number of experiments following an `identical twin' approach \citep{wang_data_2015}. The agent-based model is first executed to produce hypothetical real data -- also known as pseudo-truth \citep{grazzini_bayesian_2017} data -- and these data are assumed to be drawn from the real world. During data assimilation, observations are derived from the pseudo-truth data. This approach has the advantage that the `true' system state can be known precisely, and so the accuracy of the particle filter can be calculated. In reality, the true system state can never be known.

The agent-based model under study is designed to represent a very simple pedestrian system. It has been kept intentionally simple because the aim here is to experiment with the particle filter, not to accurately simulate a pedestrian system. Were the model more complicated it would become more difficult to understand the internal uncertainties, which would in turn make it more difficult to understand how well the particle filter was able to handle these uncertainties. The model is sufficiently complex to allow the emergence of crowding, so its dynamics could not be easily replicated by a simpler mathematical model -- as per \citep{lloyd_exploring_2016} and \citep{ward_dynamic_2016} -- but is otherwise as simple as possible. Crowding occurs because the agents have a variable maximum speed, therefore slower agents hold up faster ones who are behind them. The only uncertainty in the model, which the particle filter is tasked with managing, occurs when a faster agent must make a random choice whether to move round a slower agent to the left or right. Without that uncertain behaviour the model would be deterministic. A more realistic crowd simulation \citep[e.g.][]{helbing_simulating_2000} would exhibit much more complicated behavioural dynamics.

The paper is outlined as follows: Section~\ref{background} reviews the relevant literature; Section~\ref{method} outlines the methods, including a description of the agent based model and particle filter; Section~\ref{experiments} outlines the experiments that are conducted and their results; and Section~\ref{discussion} draws conclusions and outlines opportunities for future work.

% !TEX root = ParticleFilter.tex
\section{Background\label{background}}

\begin{itemize}
\item How people have tried to do state (and parameter?) estimation in ABMs before
\item Difference between normal parameter estimation with a (e.g.) GA and dynamic state estimation
\item Data assimilation methods, focussing on Particle Filter
\item Data assimilation methods in ABM (will be brief!)
\end{itemize}

% !TEX root = ParticleFilter.tex
\section{Method\label{method}}


\subsection{The Agent-Based Model: StationSim}

\textit{StationSim} is a simple agent-based model that has been designed to very loosely represent the behaviour of a crowd of people moving from an entrance on one side of a rectangular environment to an exit on the other side. This is analogous to a train arriving at a train station and passengers moving across the concourse to leave. A number of agents, $N$, which varies in the later experiments, are created when the model starts. They are able to enter the environment (leave their train) at a uniform rate through one of three entrances. They move across the `concourse' and then leave by one of the two exits. The entrances and exits have a set size, such that only a limited number of agents can pass through them in any given iteration. Once all agents have entered the environment and passed through the concourse then the simulation ends. The model environment is illustrated in Figure \ref{fig:StationSim}, with the trajectories of two interacting agents for illustration. 
%The model is also outlined in full as per the ODD protocol~\citep{grimm_odd_2010} in Appendix~\ref{odd}.

\begin{figure}[ht]
\centering
\includegraphics[width=0.5\textwidth]{figures/PF_ABM}
\caption{StationSim environment with 3 entrance and 2 exit doors}.\label{fig:StationSim}
\end{figure}

The model has deliberately been designed to be extremely simple and does not attempt to match the behavioural realism offered by more developed crowd models \citep{chen_multiagentbased_2017, helbing_simulating_2000, klugl_largescale_2007, vanderwal_simulating_2017}. The reason for this simplicity is so that: (1) the model can execute relatively quickly; (2) the probabilistic elements in the model are limited (we know precisely from where probabilistic behaviour arises); (3) the model can be described fully using a relatively simple state vector, as discussed in Section~\ref{state_vector}. Importantly, the model is able to capture the emergence of \textit{crowding}. This results because each agent has a different maximum speed that they can travel at. Therefore, when a fast agent approaches a slower one, they attempt to get past by making a random binary choice to move left or right around them. Depending on the agents in the vicinity, this behaviour can start to lead to the formation of crowds. To illustrate this, Figure~\ref{fig:crowding} shows the paths of the agents (\ref{fig:crowding-trails}) and the total agent density (\ref{fig:crowding-density}) during an example simulation.   The degree and location of crowding depends on the random allocation of maximum speeds to agents and their random of direction taken to avoid slower agents; these cannot be estimated \textit{a priori}. Unlike in previous work where the models did not necessarily meet the common criteria that define agent-based models \citep[e.g.]{lloyd_exploring_2016, ward_dynamic_2016} this model respects three of the most important characteristics: 

\begin{figure}
    \centering
    \begin{subfigure}[b]{0.48\textwidth}
        \centering
        \includegraphics[width=.95\linewidth]{figures/crowding-trails}
        \caption{Individual trails showing the paths taken by agents}
        \label{fig:crowding-trails}
    \end{subfigure}
    \label{fig:crowding}
    \begin{subfigure}[b]{0.48\textwidth}
        \centering
        \includegraphics[width=.95\linewidth]{figures/crowding-density}
        \caption{The total crowd density over the simulation run.}
        \label{fig:crowding-density}
    \end{subfigure}
    \caption{An example of crowding in the StationSim model}
    \label{fig:crowding}
\end{figure}

\begin{itemize}
	\item individual heterogeneity -- agents have different maximum travel speeds; 
	\item agent interactions -- agents are not allowed to occupy the same space and try to move around slower agents who are blocking their path; 
	\item emergence -- crowding is an emergent property of the system that arises as a result of the choice of exit that each agent is heading to and their maximum speed.
\end{itemize}

The model code is relatively short and easy to understand. It is written in Python, and is available in its entirety at in the project repository \citep{stationsimgit}.
% \footnote{\url{XXXX Link to repo}}.


\subsection{Data Assimilation - Introduction and Definitions}

DA methods are built on the following assumptions: 

\begin{enumerate}
	\item Although they have low uncertainty, observational data are often spatio-temporally sparse. Therefore there are typically insufficient amounts of data to to describe the system in sufficient detail and a data-driven approach would not work.
	\item Models are not sparse; they can represent the target system in great detail and hence fill in the spatio-temporal gaps in observational data by propagating data from observed to unobserved areas \citep{carrassi_data_2018}. For example, some parts of a building might be more heavily observed than others, so a model that assimilated data from the observed areas might be able to estimate the state of the unobserved areas. However, if the underlying systems are complex, a model will rapidly diverge from the real system in the absence of \textit{up to date} data \citep{ward_dynamic_2016}.
	\item The combination a model and up-to-date observational data allow ``all the available information'' to be used to determine the state of the system as accurately as possible \citep{talagrand_use_1991}. 
\end{enumerate}

DA algorithms work by running a model forward in time up to the point that some new observational data become available. This is typically called the \textit{predict} step. At this point, the algorithm has an estimate of the current system state and its uncertainty (the prior). The next step, \textit{update},  involves using the new observations, and their uncertainties, to update the current state estimate to create a posterior estimate of the state. As the posterior has combined the best guess of the state from the model \textit{and} the best guess of the state from the observations, it should be a closer estimate of the true system state than that which could be estimated from the observations or the model in isolation.

\subsection{The Particle Filter\label{particle_filter}}

There are many different ways to perform data assimilation, as discussed in Section \ref{da_pf}. Here, a potentially appropriate solution to the data assimilation problem for agent-based models is the particle filter -- also known as a Bayesian bootstrap filter or a sequential Monte Carlo method -- which represents the posterior state using a collection of model samples, called particles \citep{gordon_novel_1993,carpenter_improved_1999,wang_data_2015, carrassi_data_2018}. Figure~\ref{fig:PF_flowchart} illustrates the process of running a particle filter. Note that the `pseudo-truth model' is a single instance of the agent-based model that is used as a proxy for the real system as per the identical twin experimental framework that we have adopted.

The data assimilation `window' determines how often new observations are assimilated into the particle filter. The size of the window is an important factor -- larger windows result in the particles deviating further from the real system state -- but here we fix the window at 100 iterations. The simulation terminates when all agents have left the system.

\begin{figure}[ht]
\centering
\includegraphics[width=0.8\textwidth]{figures/PF_flowchart2}
\caption{Flowchart of data assimilation process using a particle filter.\label{fig:PF_flowchart}}
\end{figure}

\subsubsection{The State Vector and Transition Function\label{state_vector}} 

Here, the \textit{state vector}, at a time $t$, contains all the information that a transition function needs to iterate the model forward by one step, including all of the agent ($i = \{ 0, 1, \dots, N \} $) parameters ($\overrightarrow{p_i}$) and variables ($\overrightarrow{v_i}$) as well as global model parameters $\overrightarrow{P}$:
\begin{equation}
  S_t  = \left[ \begin{array}{cccccccc}
\overrightarrow{p_0} & \overrightarrow{v_0} & \overrightarrow{p_1} &  \overrightarrow{v_1} &  \dots &  \overrightarrow{p_N} &  \overrightarrow{v_N} & \overrightarrow{P} 
\end{array} \right]
\end{equation} 

A similar structure, the \textit{observation vector}, contains all of the observations made from the `real world' (in this case the pseudo-truth model) at a time $t$. Here, the particle filter is only used to estimate the state of the models variables ($\overrightarrow{v_i}$), not any of the parameters ($\overrightarrow{p_i}$ and $\overrightarrow{P}$) (although it is worth noting that parameter estimation is technically feasible and will be experimented with in future work). Also, the current speed of an agent can be calculated from its current location and the locations of the agents surrounding it, so in effect the observation vector only needs to include the positions of the agents with the addition of some Gaussian noise, $\epsilon$:
\begin{equation}
  O_t  = \left[ \begin{array}{ccccccc}
x_0 & y_0 & x_1 & y_1 & \dots & x_n & y_n 
\end{array} \right]
\end{equation} 

 Therefore in the experiments conducted here, all model parameters are fixed. Hence a further vector is required to map the observations to the state vector that the particles can actually manipulate. We define the partial state vector $S'$ to match the shape of $O$, i.e.:
\begin{equation}
  S'_t  = \left[ \begin{array}{ccccccc}
x_0 & y_0 & x_1 & y_1 & \dots & x_n & y_n 
\end{array} \right]
\end{equation} 

This has the effect of `pairing' agents in the particles to those in the pseudo-truth data, in a similar approach to that taken by \citep{wang_data_2015}. It is worth noting that, because the particle filter will not be tasked with parameter estimation, then the data assimilation is somewhat simpler than it would be in a real application. For example, one of the agent parameters is used to store the location of the exit out of the environment that the agent is moving towards. As this parameter is set \textit{a priori} for each agent, then the particle filter does not need to estimate where the agents are ultimately going, only where they currently are.

\subsubsection{Observations from the pseudo-truth data}

In a real application, the particle filter would be assimilating data in real time from sensors of the real world. This is not the case here so instead, we take `observations' from the pseudo-truth data (which are, as it happens, generated by the StationSim model). Each particle evolves in time according to the StationSim dynamics and receives new observations at regular intervals. Measurement error (i.e. noise, $\epsilon$) is added to the observations (in the real world, sensors observations will be noisy). Here, observations take the form of the $(x,y)$ locations of all agents. This is analogous to tracking all individuals in a crowd and providing snapshots of their locations at discrete points in time to the particle filter. This `synthetic observation' is probably more detailed than reality, so future work will vary the amount of detail provided to the algorithm. It will firstly reduce the number of agents who are observed (e.g. tracking only \textit{some} people) and then provide only aggregate population counts (which is analogous to using a camera or other sensor to count the number of people at a certain point). 

\subsubsection{Particle Weights}

Each particle in the particle filter has a weight associated with it that quantifies how similar a particle is to an observation. The weights are calculated at the end of each data assimilation window (i.e. when observations become available). At the start of the following window the particles then evolve independently from each other \citep{fearnhead_particle_2018}.  The weights are, in effect, the average distance between agents in that particle and the corresponding agents in StationSim (recall that there is a one-to-one mapping between agents in the particles and agents in the truth model). Formally, let $x^n(i,t)$ be the location of the $i$-th agent at time $t$ in the $n$-th particle for $n \in \{1,\dots,N\}$ and let $x(i,t)$ be the location of the $i$-th agent in StationSim for $i \in \{1,\dots,I\}$. The error of the $n$-th particle $\epsilon^n(t)$ at time $t$ is then given by
\begin{equation}
\epsilon^n(t) = \frac{1}{I} \sum_{i=1}^I |x(i,t) - x^n(i,t)|,
\end{equation}
and the particle filter error $\nu(t)$ at time $t$ is given by
\begin{equation}
\label{eqn:particle_error}
\begin{split}
\nu(t) =& \frac{1}{N} \sum_{n=1}^{N} \epsilon^n(t), \\
=& \frac{1}{NI} \sum_{n=1}^{N}\sum_{i=1}^I |x(i,t) - x^n(i,t)|.
\end{split}
\end{equation}

It is worth noting that, because the agent locations are the only data stored in the partial state vector, the particle error is equivalent to the Euclidean distance ($l_2$-norm) between the particle partial state vector $S'_{t}$ and the observation vector $O_t$,
\begin{equation}
 \nu(t) = || S'_{t} - O_t  ||_2.
\end{equation} 

Particles with relatively large error are likely to be removed during the sampling procedure (discussed in the following section), whereas those with low error are likely to be duplicated. In addition for their use in resampling, the population of particle weights can be used to gain insight into how well the particle filter is able to represent the `true' system state overall. 

\subsubsection{Sampling Procedure\label{particle_sampling}}

Here, a bootstrap filter is implemented which uses systematic resampling \citep{doucet_introduction_2001, douc_comparison_2005, wang_data_2015, long_spatial_2017, carrassi_data_2018}. This begins by taking a random sample $U$ from the uniform distribution on the interval $[0,1/N]$ and then selecting $N$ points $U^i$ for $i \in \{1,\dots,N\}$ on the interval $[0,1]$ such that
\begin{equation}
U^i = (i-1)/N + U.
\end{equation}
Let the particles currently have locations $x_i$. Using the inversion method, we calculate the cumulative sum of the normalised particle weights $w_i$ and define the inverse function $D$ of this cumulative sum, that is:
\begin{equation}
D(u) = i \text{ for } u \in \left(\sum_{j=1}^{i-1}w_j,\sum_{j=1}^{i}w_j\right].
\end{equation}
Finally, the new locations of the resampled particles are given by $x_{D(U^i)}$.

As discussed in Section~\ref{background}, a well-studied issue that particle filters face is that of particle deprivation \citep{snyder_obstacles_2008}, which refers to the problem of particles converging to a single point such that all particles, but one, vanish \citep{kong_sequential_1994}.  
% This vastly reduces the size of the state space covered by the population of particles and will make it difficult or impossible to find particles with low error in later windows. Common approaches to resolve the deprivation problem include increasing the number of particles or trying to reduce the dimensionality of the state space. Bespoke approaches also exist; for example \citep{wang_data_2015} develop a technique called `component set resampling' that samples \textit{parts} of particles (i.e. those parts that are working well) rather than whole particles in their entirety.  
Here, the problem is addressed in two ways. Firstly by simply using large numbers of particles relative to the size of the state space and, secondly, by diversifying the particles \citep{vadakkepat_improved_2006} -- also known as roughening, jittering, and diffusing \citep{li_fight_2014, shephard_learning_2009, pantrigo_combining_2005}. In each iteration of the model we add Gaussian white noise to the particles' state vector to increase their variance, which increases particle diversity. This encourages a greater variety of particles to be resampled and therefore makes the algorithm more likely to represent the state of the underlying model. This method is a special case of the resample-move method presented in \citep{gilks_following_2001}. The amount of noise to add is a key hyper-parameter -- too little and it has no effect, too much and the state of the particles moves far away from the true state -- as discussed in the following section. 

% !TEX root = ParticleFilter.tex
\section{Experiments\label{experiments}}

\subsection{Experiments with Uncertainty}

Purpose here is basically to see how the particle filter behaves when we give it l

\begin{enumerate}
\item Randomness in particles
\item Measurement noise (external)
\item Internal randomness (e.g. in agent behaviour)
\item (Simultaneous combinations of different randomness)
\end{enumerate}

\subsection{Experiments with Measurement Noise}

\begin{enumerate}
\item Reduce the amount of information given to the particle filter (e.g. only allow it to optimise half of the state vector).
\item Aggregate the measurements (e.g. counts per area rather than individual traces).
\end{enumerate}


% !TEX root = ParticleFilter.tex
\section{Discussion and Future Work\label{discussion}}

\subsection{Discussion}

This paper has experimented with the use of a sequential importance resampling (SIR) particle filter (PF) as a means of dynamically incorporating data into a simple agent-based model of a pedestrian system. The results demonstrate that it is possible to use a particle filter to perform dynamic adjustment of the model. However, they also show that (as expected~\citep{rebeschini_can_2015, carrassi_data_2018}) as the dimensionality of the system increases, the number of particles required to maintain an acceptable approximation error grows exponentially. The reason for this is because, as the dimensionality increases, it becomes less likely that an individual particle will have the `correct combination' of values~\citep{long_spatial_2017}. In this work, the dimensionality is proportional to the number of agents. At most 10,000 particles were used, which was sufficient for a simulation with 30-40 agents. However, for a more complex and realistic model containing hundreds or thousands of agents, the number of particles required would most likely number in the millions. The particle filter used in this study was provided with more information than would normally be available. For example, information was supplied as fixed parameters on when each agent entered the simulation, their maximum speeds, and their chosen destinations. Therefore the only information that the particle filter was lacking was the actual locations of the agents and whether they would chose to move left or right to prevents a collision with another agent. It is entirely possible to include these agent-level parameters in the state vector, but this would further increase the size of the state space and hence the number of particles required. This is an important caveat as in a real-world situation it is very unlikely that such detail would be available. Future work should begin to experiment with the number of particles that would be required when observational data that are more akin to those available in the real world are used.

There are a number of possible improvements that could be made to the basic SIR particle filter to reduce the number of particles required. For example, \citep{wang_data_2015} propose \textit{component set resampling} -- details below -- but exploring these further is beyond the scope of this paper. Overall, these results reflect those of other studies which demonstrate that particle filtering has value for simulations with relatively few agents and interactions \citep[e.g.][]{wang_data_2015, lueck_who_2019, kieu_dealing_2019} but that the large dimensionality of a pedestrian system poses problems for the standard (unmodified) bootstrap filter. 

\subsection{Improvements to the particle filter}

There are a number of adaptions to the main particle filtering algorithm that might make the method more amenable to use with complex agent-based models. The aforementioned Component Set Resampling \citep{wang_data_2015} approach proposes that individual components of particles are sampled, rather than whole particles in their entirety. A more commonly used approach is to reduce the dimensionality of the problem in the first place. With spatial agent-based models, such as the one used here, spatial aggregation provides such an opportunity. In the data assimilation stage, the state vector could be converted to an aggregate form, such as a population density surface, and particle filtering could be conducted on the aggregate surface rather than on the individual locations of the agents. After assimilation, the particles could be disaggregated and then run as normal. This will, of course, introduce error because the exact positions of agents in the particles will not be known when disaggregating, but that additional uncertainty might be outweighed by the benefits of a more successful particle filtering overall. In addition, the observations that could be presented to an aggregate particle filter might be much more in line with those that are available in practice (as will be discussed shortly). If the aim of real-time model calibration is to give decision makers a \textit{general idea} about how the system is behaving in real time, then this additional uncertainty might not be problematic. A similar approach, proposed by \citep{rebeschini_can_2015}, could be to divide up the space into smaller regions and then apply the algorithm locally to these regions (i.e. a divide-and-conquer approach). Although it is not clear how well divide-and-conquer would work in an agent-based model -- for example, \citep{long_spatial_2017} developed the method for a discrete cellular automata model -- it would be an interesting avenue to explore.

\subsection{Real-World Implications and Future Work}

It is important to note that, unlike other data assimilation approaches, the particle filter does not dynamically alter the state of the running model. This could be advantageous because, with agent-based modelling, it is not clear that the state of an agent should be manipulated by an external process. Agents typically have goals and a history, and behavioural rules that rely on those features, so artificially altering an agent's internal state might disrupt their behaviour making it, at worst, nonsensical. Experiments with alternative (potentially more efficient) algorithms such as 4DVar or a the Ensemble Kalman Filter should be conducted to test this. 

Ultimately the aim of this work is to develop methods that will allow simulations of human pedestrian systems to be optimised in real time. Not only will such methods provide decision makers with more accurate information about the present, but they could also allow for better predictions of the near future \citep[c.f.][]{kieu_dealing_2019}. One assumption made throughout the paper, which limits its direct real-world use, is that the locations of pseudo-real individuals are known, albeit with some uncertainty. Not only is this assumption unrealistic -- it is rare for individuals to be tracked to this degree in public places -- but we would also argue that the privacy implications of tracking individual people are not outweighed by the benefits offered by better understanding the system. Therefore, immediate future work will test how well a data assimilation algorithm would fare were it supplied only aggregate information such as the number of people who pass through a barrier, or the number of people recorded by a CCTV camera within a particular area. Both of these measures can be used in aggregate form and would be entirely anonymous. It is unclear whether such aggregate data would be sufficient to identify a `correct' agent-based model, so experiments should explore the spatio-temporal resolution of the aggregate data are required. Also, identifiability/equifinality analysis might help initially as a means of estimating whether the available data are sufficient to identify a seemingly `correct' model in the first place. In the end, such research might help to provide evidence to policy makers for the number, and characteristics, of the sensors that would need to be installed to accurately simulate the target system, and how these could be used to maintain the privacy of the people who they are recording.






%\bibliographystyle{splncs04}
%\bibliographystyle{chicago}
\bibliographystyle{unsrt}
\bibliography{2018-ParticleFilter}

\begin{appendices}
    \section{Instructions for Running the Source Code\label{appendix:code}}

The source code to run the StationSim model and the particle filter experiments can be found in the main `\href{https://dust.leeds.ac.uk/}{Data Assimilation for Agent-Based Modelling}' (DUST) project repository:
\begin{quote}
    \texttt{
        \href{https://github.com/urban-analytics/dust}
        {github.com/urban-analytics/dust}
    }
\end{quote}

Specifically, scripts and instructions to run the experiments are available at: 
\begin{quote}
    \texttt{
        \href{https://github.com/Urban-Analytics/dust/tree/master/Projects/ABM_DA/experiments/pf_experiments}
        {github.com/Urban-Analytics/dust/tree/master/Projects/ABM\_DA/experiments/pf\_experiments}
    }
\end{quote}

\end{appendices}

\end{document}
