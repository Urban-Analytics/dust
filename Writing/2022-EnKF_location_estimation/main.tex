\documentclass{article}

% Packages
\usepackage{amsmath}
\usepackage{graphicx}
\usepackage[round]{natbib}

% Document metadata
\title{Untitled EnKF paper}
\author{Keiran Suchak}

\begin{document}

\maketitle{}

\begin{abstract}
    Write abstract here.
\end{abstract}

\section{Introduction}\label{sec:intro}

Contents:
\begin{itemize}
    \item Provide context and motivation for investigation.
    \item Outline aims and objectives.
\end{itemize}

Points of distinction to highlight:
\begin{itemize}
    \item Defining an approach for defining whether an agent is active or
        inactive in an ensemble of models.
    \item Comparing error in ensemble mean with mean of errors of
        ensemble-member models.
    \item Explaining the importance of an appropriate summary statistic
        (median instead of mean) when calculating the average error over
        time.
    \item Explaining the importance of considering time-steps when a
        sufficient number of filters are still running when collecting
        summary statistics of multiple filter runs.
    \item Using EnKF to improve the accuracy with which an ABM simulates a
        pedestrian system.
\end{itemize}

\section{Background}\label{sec:background}

\begin{itemize}
    \item Discuss previous relevant work:
    \begin{itemize}
        \item \citet{ward2016dynamic}
        \item \citet{malleson2020simulating}
        \item \citet{clay2020towards}
    \end{itemize}
\end{itemize}

\section{Methods}\label{sec:methods}

\subsection{Model}\label{sub:methods:model}

Explain about \texttt{StationSim\_GCS}.

\subsection{Ensemble Kalman Filter}\label{sub:methods:enkf}

\begin{itemize}
    \item Explain about the Ensemble Kalman Filter~\citep{evensen2003ensemble},
        which is based on the Kalman Filter~\citep{kalman1960new}.
\end{itemize}

\section{Experiments}\label{sec:exp}

\subsection{Benchmarking}\label{sub:exp:bench}

\subsection{Untitled Section}

\begin{itemize}
    \item Talk about measures used when running experiments with multiple EnKFs
        to ensure that outliers don't skew results:
    \begin{itemize}
        \item Median instead of mean error.
        \item Only considering time-steps when a sufficient number of models are
            active.
    \end{itemize}
\end{itemize}

\section{Results}\label{sec:results}

\subsection{Benchmarking}\label{sub:results:bench}

\section{Conclusion}\label{sec:conc}

\bibliographystyle{unsrtnat}
\bibliography{references}

\end{document}
